\documentclass[11pt, oneside]{article}   	% use "amsart" instead of "article" for AMSLaTeX format
\usepackage{geometry} 
\usepackage{color}               		% See geometry.pdf to learn the layout options. There are lots.
\geometry{letterpaper}                   		% ... or a4paper or a5paper or ... 
%\geometry{landscape}                		% Activate for rotated page geometry
%\usepackage[parfill]{parskip}    		% Activate to begin paragraphs with an empty line rather than an indent
\usepackage{graphicx}				% Use pdf, png, jpg, or eps§ with pdflatex; use eps in DVI mode
								% TeX will automatically convert eps --> pdf in pdflatex		
\usepackage{amssymb}
\usepackage{setspace}
\usepackage{amsmath}

\usepackage{appendix}   % Start an appendices environment, then treat each separate appendix as an ordinary section

\usepackage{authblk}

\usepackage[semicolon]{natbib}
\usepackage{verbatim}
\usepackage{soul}
\usepackage{graphicx}
\usepackage{subcaption}
\usepackage{hyperref}
\def\UrlBreaks{\do\/\do-}

\usepackage{tikz}	% For flow charts
\usetikzlibrary{shapes, arrows}
\tikzstyle{compartment} = [circle, minimum width=2cm, text centered, draw=black, fill=none]
\tikzstyle{arrow} = [thick,->,>=stealth]

\usepackage{multirow}

\begin{document}

Consider an $SIS$-type disease with two strains, call them $I$ and $J$. Assume also that we have a vaccine for this disease which is fully effective against strain $J$ and partially effective against strain $I$. If we include demography, the dynamics are as follows:
%
\begin{align}
    S' &= \Lambda - \mu S - \tau_I S I - \tau_J S J + \chi \gamma_I I + \gamma_J J - \psi S \label{eq:S_prime}\\
    V' &= \psi S - \mu V - \delta \tau_I V I + (1 - \chi) \gamma_I I \label{eq:V_prime}\\
    I' &= \tau_I S I + \delta \tau_I V I - \mu I - \gamma_I I \label{eq:I_prime}\\
    J' &= \tau_J S J - \mu J - \gamma_J J \label{eq:J_prime}
\end{align}
%
Parameters are defined in Table \ref{tab:Parameters}.

\begin{table}
\centering
\begin{tabular}{|c|c|}
    \hline
    $\Lambda$ & Birth rate\\
    \hline
    $\mu$ & Death rate\\
    \hline
    $\psi$ & Vaccination rate\\
    \hline
    $\tau_I$, $\tau_J$ & Infection rates\\
    \hline
    $\gamma_I$, $\gamma_J$ & Recovery rates\\
    \hline
    $\delta$ & Relative susceptibility of vaccinated individuals to infection with strain $I$\\
    \hline
    $\chi$ & Proportion of strain $I$ infections who recover to compartment $S$\footnotemark\\
    \hline
\end{tabular}
\caption{Parameter definitions for our two-strain SIS model with vaccination and demography.}
\label{tab:Parameters}
\end{table}

\footnotetext{The $\chi$ parameter can be thought of as the combined effect of what proportion of the population is vaccinated and what fraction of strain $I$ infections produce antibodies which are functionally equivalent to the vaccine.}

The disease-free equilibrium can be computed in the usual way, giving $S_0 = \Lambda / (\mu + \psi)$ and $V_0 = \psi S_0 / \mu$. Note that the equilibrium population size is always $\Lambda / \mu =: N_*$ so, as we would expect, $S_0 + V_0 = N_*$. A local stability analysis gives the following two $\mathcal{R}_0$-type values: \hl{The values in the book assume $\Lambda = \mu$. Since I'm trying to avoid this assumption so I can see what role is played by the population size, I need to re-derive these quantities. Tldr: The following quantities are wrong and need to be re-derived.}
%
\begin{align*}
    \mathcal{R}_0^I &= \frac{\tau_I (\mu + \delta \psi)}{(\mu + \gamma_I) (\mu + \psi)}\\
    \mathcal{R}_0^J &= \frac{\tau_I (\mu + \delta \psi)}{(\mu + \gamma_J) (\mu + \psi)}    
\end{align*}

We are actually more interested in the endemic equilibria. To this end, we now assume that $I, J \neq 0$. This assumption allows us to cancel $I$ and $J$ in equations \ref{eq:I_prime} and \ref{eq:J_prime}. \hl{More to come.}

\end{document}